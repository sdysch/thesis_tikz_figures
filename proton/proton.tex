\begin{tikzpicture}
	
	% origin for main proton circle
	\newcommand{\xreference}{0}
	\newcommand{\yreference}{0}
	\newcommand{\protonradius}{4cm}
	\newcommand{\quarkradius}{5cm * 0.15}
	\newcommand{\seaquarkradius}{\quarkradius * 0.4}
	\newcommand{\offset}{2}
	\newcommand{\ylowerquark}{0.73205080757 * \offset}

	% proton
	\filldraw[blur shadow={shadow blur steps=5}, color=gray!20, draw=black] (\xreference, \yreference + 1) circle (\protonradius);


	%% labels
	%\newcommand{\labeloffset}{0.25}
	%\node at (\xreference + \offset + \labeloffset, \yreference + \offset + \labeloffset) {\color{white}\Large{u}};
	%\node at (\xreference - \offset - \labeloffset, \yreference + \offset + \labeloffset) {\color{white}\Large{d}};
	%% from Pythagorous: sqrt{3}x - y
	%\node at (\xreference + 0, 3 * \xreference - \yreference) {\color{white}\Large{d}};
	
	\draw[gluon] (\xreference + \offset, \yreference + \offset) -- (\xreference - \offset, \yreference + \offset);
	\draw[gluon] (\xreference + \offset, \yreference + \offset) -- (\xreference, -1 * \ylowerquark);
	\draw[gluon] (\xreference - \offset, \yreference + \offset) -- (\xreference, -1 * \ylowerquark);


	% gluons for sea quarks
	\newcommand{\gluonoffset}{1.5}
	\draw[gluon] (\xreference + \offset, \yreference + \offset) -- (\xreference + \offset - \gluonoffset, \yreference + \offset + \gluonoffset);
	\draw[gluon] (\xreference - \offset, \yreference + \offset) -- (\xreference - \offset + \gluonoffset, \yreference + \offset + \gluonoffset);

	\draw[particleNoArrow] (\xreference + \offset - \gluonoffset, \yreference + \offset + \gluonoffset) -- (\xreference + \offset - \gluonoffset - 0.5, \yreference + \offset + \gluonoffset + 0.5);
	\draw[particleNoArrow] (\xreference + \offset - \gluonoffset, \yreference + \offset + \gluonoffset) -- (\xreference + \offset - \gluonoffset - 0.5, \yreference + \offset + \gluonoffset - 0.5);

	\draw[particleNoArrow] (\xreference - \offset + \gluonoffset, \yreference + \offset + \gluonoffset) -- (\xreference - \offset + \gluonoffset + 0.5, \yreference + \offset + \gluonoffset + 0.5);
	\draw[particleNoArrow] (\xreference - \offset + \gluonoffset, \yreference + \offset + \gluonoffset) -- (\xreference - \offset + \gluonoffset + 0.5, \yreference + \offset + \gluonoffset - 0.5);


	% draw quarks last to "hide" gluons that overlap in centre
	% valence quarks
	\shadedraw[inner color=blue!50, outer color=blue, draw=black] (\xreference + \offset, \yreference + \offset) circle (\quarkradius);
	\shadedraw[inner color=red!50, outer color=red, draw=black] (\xreference - \offset, \yreference + \offset) circle (\quarkradius);
	% from Pythagorous: down offset is sqrt(3) * x offset - y offset = (sqrt(3) - 1) * offset
	\shadedraw[inner color=red!50, outer color=red, draw=black] (\xreference, -1 * \ylowerquark) circle (\quarkradius);

	% sea quarks
	\shadedraw[inner color=orange!50, outer color=orange, draw=black] (\xreference, \yreference + 2 * \offset) circle (\seaquarkradius);
	\shadedraw[inner color=orange!50, outer color=orange, draw=black] (\xreference, \yreference + 1.5 * \offset) circle (\seaquarkradius);

	% labels
	\node at (\xreference + \offset, \yreference + \offset) {\Large{\color{white}d}};
	\node at (\xreference - \offset, \yreference + \offset) {\Large{\color{white}u}};
	\node at (\xreference, -1 * \ylowerquark) {\Large{\color{white}u}};

\end{tikzpicture}
